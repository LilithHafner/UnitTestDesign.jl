
% JuliaCon proceedings template
\documentclass{juliacon}
\setcounter{page}{1}
\usepackage{xspace}
\newcommand{\utd}{\texttt{UnitTestDesign}\xspace}

\begin{document}

\input{header}

\maketitle

\begin{abstract}
Combinatorial interaction testing is an automated way to generate test cases for unit tests. It's designed to be the best guess at the fewest unit tests that will give the best coverage. This article discusses when to use this technique and how to apply it with the \utd package in the Julia testing ecosystem. This article offers a general approach to using automated test generation for different software testing applications.
\end{abstract}

\section{Introduction}

We test code in order to address risk of failure, and we judge risk of failure by the likelihood code will fail, how difficult it is to mitigate a failure, and how important it is to have a correct result. However, whether the result of a calculation matters less than what's in the environment~\cite{Wiklund2017-ms}.
\vskip 6pt
Our first unit tests for code often derive from mimicking how a user might write client code. These happy-path tests are narrative. We may augment them with our best guesses for corner cases of a function under test. There are times we want more of a guarantee that a function doesn't have faults. This may be that we recognize the function is complex, or that it's central to a sensitive calculation. We may test a part of a code base well because we want reduce the logical complexity that comes from uncertainty~\cite{Sha2001-ie}.

\vskip 6pt
There is a world of testing tools to help find faults in risky code. Some measure the quality of existing tests. Others select tests to run depending on test coverage or recent code modifications. Let's focus on the first antidote to the biases inherent in narrative testing, those that generate test cases.

\vskip 6pt
\begin{lstlisting}[language=Julia]
using UnitTestDesign

@enum Strategy naive keyfitz greville monotonic
arg1 = [1, 4, 8]
arg2 = [:a, :b, :c, :d]
arg3 = [naive, keyfitz, greville, monotonic]
arg4 = ["none", "RK4", "quadrature", "QUADPACK"]
arg5 = [0.0, 9.3, -1.2, Inf]

test_cases = all_pairs(arg1, arg2, arg3, arg4, arg5)
for test_case in test_cases:
    result = samplefunction(test_case...)
    @assert known_invariant(result, test_case)
end
\end{lstlisting}

Here, the \verb|all_pairs| function uses the sample argument values to create a set of test cases. The question is which set of cases it creates and when they are appropriate.

\vskip 6pt
There are a few common strategies for automated test case generation. A randomized approach chooses argument values from the space of allowed arguments, usually with some bias towards choosing possible corner cases~\cite{Lampropoulos2020-sd,Arcuri2012-az}. Some tools introduce more structure to choosing random arguments. The \texttt{QuickCheck} and \texttt{Hypothesis} packages use customized generators to create streams of test cases~\cite{loscher2018automating}. Concolic testing records the execution of the function under test in order to generate subsequent test cases that are likely to increase line coverage~\cite{King1976-jt,Wang2018-xh,vira2019}. There are also more mathematical approaches, such as Sobol sequences and orthogonal arrays~\cite{He2013-th}, which have particular diagnostic uses.


\section{Statement of Need}\label{sec:statement-of-need}

Most of the techniques for test case generation focus more on making a smart set of tests than they focus on making few tests that are of high quality. Combinatorial testing generates fewer tests that are designed to give good branch coverage~\cite{Nie2011-yl,Grindal2005-su,Kuhn2010-ak}.

\vskip 6pt
There are two cases where this is crucial. The first is that you have to test a slow function, such as a large, Monte Carlo inference. If the function under test runs for hours or days, then selective testing can respect resource limitations.

\vskip 6pt
The other case is when there is a large test space. We've been talking about a function under test, but this is a stand-in for any test we can parametrize. Let's say we've written an application, that we've unit tested its important supporting functions, and we want to write user-level tests that look for problems from interactions among command-line arguments and choices in parameter files. This can be equivalent to testing a function with twenty or more arguments. Twenty arguments, with four possible values each, would lead to over $10^{12}$ ways to call this function, so how can we feel our tests begin to cover the problem?

\vskip 6pt
The combinatorial testing in \utd takes the time to optimize the choice of test cases so that they can test the code well. The theory of the game is that each if--then in the code will branch depending on combinations of argument values, so combinatorial testing ensures all combinations of $n$ arguments are tried in at least one test case. Here, $n=2$ for two-way testing, also known as all-pairs testing~\cite{pairwise-website}, but studies have shown that wayness up to $n=6$ can be useful~\cite{Petke2015-ex}.

\vskip 6pt
While combinatorial testing is a useful tool in some circumstances, there are other tools that generate combinatorial tests. X makes them with a domain-specific language. This company's tool does it.

\vskip 6pt
These companies keep disappearing. The quality of tools is supect (by the authors sometimes). It comes down to an author using the tools available within a framework.

\vskip 6pt
That means the tool has to be usable, too. What happens if we create an all-pairs set of unit tests, but some of them don't work for your function because the combination of arguments doesn't make sense? You have to throw out those tests, but you've lost the very guarantee that this testing method made.
So it has to have a) forbid combinations of argument values, b) n-way testing c) multi-way testing d) start with certain cases~\cite{Czerwonka2006-hm}.


\section{How to Use Combinatorial Testing}\label{sec:how-to-use}

\subsection{Outline of the method}

While you may already know how to use test case generation from the snippet in the introduction, it can help to think of automated test case generation as a set of steps, each of which has choices to make.
\begin{enumerate}
   \item Identify the system under test. It may be a function, but it could correspond to an \textsc{api} or an application.
   \item Decide how a single test case, consisting of argument values, should correspond to a unit test.
   \item Address the risk associated with different arguments.
   \item Pick a test generation strategy.
   \item Choose a method for checking the results of each test.
\end{enumerate}
We'll explore these steps in this section.

\subsection{Identify the system under test}

We have assumed, thus far, that we are unit testing a function. In that case, the function has arguments and those arguments can take on particular values. Our test cases will correspond exactly to an invocation of the function. However, this need not be the case. The tool can be used more generally. Combinatorial testing is used outside Julia. It would apply, for instance, to testing compatibility of hardware subsystems.

A common case in Julia is testing a scientific application at the user level. Here, we can consider the user's choices of command-line arguments and parameters in files. We could even generate classes of data according to keys that we assign to arguments in \utd.

\subsection{Decide argument values}

Some arguments to a function can take on only a few allowed values, such as \verb|true| or \verb|false|, but most arguments are integers, floating-point, or more complicated data structures. The theory of covering arrays presumes any argument will be assigned only one of a few values, so we need to make choices.

\vskip 6pt
We're looking for faults in the code that lead to failures of unit tests. These faults will be visible, or invisible, for different choices of argument values. If a given function fails for \verb|arg3<2.7|, then it will fail for \verb|arg3=2| as well as \verb|arg3=2.5|, so we consider those equivalent with respect to finding faults. Our goal, as authors of unit tests, is to select argument values that represent equivalence classes for the function under test. There isn't a lot of guidance on how to choose these values. You can guess them by looking at the code or by looking at a specification for the function.

\vskip 6pt
One way to defend against our uncertainty about equivalence classes is to combine combinatorial test generation with random testing. For each argument, partition the space of values into equivalence classes. Then generate test cases, but use the values in those test cases to select randomly from the partitions. This is one way to introduce bias, and control, into random test selection. It points to treating test case generation as a continuum of techniques.

\vskip 6pt
The argument values in test cases could also represent, for instance, the number of times to call a function on an \textsc{api}. The argument values could represent choices about what arguments to pass to each function in a series of function calls.

\vskip 6pt
There are cases where a function has one argument that is a flag and, when it's chosen, other argument values don't make sense to choose. We want to forbid these from the possible input choices. For instance, if \verb|arg1| can be \verb|true| or \verb|false|, and \verb|arg2| can be \verb|default|, \verb|high|, or \verb|low|, but we want to exclude the combination of \verb|false| and \verb|high| or \verb|low|, then the possible outcomes are \verb|true|-\verb|default|, \verb|true|-\verb|high|, \verb|true|-\verb|low|, \verb|false|-\verb|default|. We can either teach the test generator to forbid two combinations, or we can modify the input values, so that \verb|arg12| represents the four possible cases for \verb|arg1| and \verb|arg2| together.

\vskip 6pt
Forbidden tuples~\cite{Petke2015-ex} for combinatorial testing is a method. Among others~\cite{Grindal2006-vy}.
\begin{lstlisting}[language=Julia]
disallow(n, level, value, kind) = level == "high" && kind == :optim
test_set = all_pairs(
    [1, 2, 3], ["low", "mid" ,"high"], [1.0, 3.7, 4.9], [:greedy, :relax, :optim];
    disallow = disallow
    )
\end{lstlisting}

\vskip 6pt
Pre-existing example calls.
\begin{lstlisting}[language=Julia]
must_test = [[1, "mid", 3.7, :relax], [1, "mid", 4.9, :relax]]
test_cases = all_pairs(
    [1, 2, 3], ["low", "mid" ,"high"], [1.0, 3.7, 4.9], [:greedy, :relax, :optim];
    seeds = must_test
    )
\end{lstlisting}

\subsection{Address risk}

The combinatorial testing approach gives you a lever with which to address perceived risk in code. You decide the wayness of coverage, which is the number of arguments for which all combinations of argument values must appear. Pairwise testing has two-way coverage. Literature on how to generate these tests uses the word coverage to mean two-way, three-way, and so on. The term test coverage usually refers to which lines of code, if--then decision tests, or branches are executed during a test.

\vskip 6pt
The two uses of the word coverage are related, because combinations with higher wayness should lead to more decision coverage of code. It would therefore seem appropriate to use cyclomatic complexity measures to estimate whether two-way coverage is enough. While there is some evidence that high coverage finds faults in less-used parts of a code base~\cite{Cai2005-ex}, coverage shouldn't be the focus of unit testing~\cite{Inozemtseva2014-gz}. This can range from pairwise testing, which is two-way coverage, to six-way coverage, beyond which studies show little benefit~\cite{Petke2015-ex}.

\vskip 6pt
As described in Sec.~\ref{sec:statement-of-need}, combinatorial testing is most useful when there are constraints on testing resources. These constraints can force choices for the wayness of coverage. There are a couple of techniques for working within constraints, while still respecting our perception of risk.

\vskip 6pt
The first is to vary wayness of coverage for different sets of arguments. For instance, if the function under test used an evolutionary algorithm that had several options for the representation of individuals, their mutation, and recombination, we might want to test these choices exhaustively with full-factorial coverage, while selecting pairwise coverage for other options. This would be done by adding a \verb|wayness| argument to the function call.

\begin{lstlisting}[language=Julia]
array_of_forty_parameters = fill(1:4, 40)
test_set = all_pairs(
    array_of_forty_parameters...;
    wayness = Dict(3 => [[3:6], [25:30]])
    )
\end{lstlisting}
It's unclear whether mixing wayness will still yield a small set of test cases because mixed-level covering arrays don't have well-understood bounds~\cite{Cohen2003-pg}, however sample runs show it does produce fewer test cases than raising coverage on all arguments.

\vskip 6pt
One of the most powerful ways to address risk while conserving resources is to pair automated test generation with automated test selection. These are methods that evaluate the tests that are generated in order to select those that are most likely to find different faults in the code.

\subsection{Pick a test generation strategy}

There are a few different strategies for combinatorial test generation in \utd.

\vskip 6pt
The \emph{full factorial} method generates $n$-way coverage for $n$ arguments. That is, it generates all combinations of values for all arguments. This is equivalent to what you could create with an $n$-deep for-loop.

\vskip 6pt
The \emph{excursion} method makes a first test case using the first provided value for each argument. Then, for one-way excursions, it walks the first argument through all values, keeping the rest of the arguments the same. Then it walks the second argument through all values, keeping the rest of the arguments the same. It asks how the function under test would perform if you were to change any single argument value. A two-way excursion tries changes to any two argument values in combination.

\vskip 6pt
There are two ways the \utd package creates covering arrays, also known as \emph{fractional factorial designs}. The default generator is called \verb|IPOG| because it is similar to the \textsc{ipo-g} algorithm~\cite{Lei2008-xt}. It is deterministic, so it produces the same set of test cases for the same set of inputs. The other generator is called \verb|GND|, and it uses a random number generator to search for covering arrays. It returns test cases that obey the same covering properties as the \verb|IPOG| generator, but it uses a method similar to the \textsc{aetg} generator~\cite{Cohen1997-lb}. In both cases, the algorithms vary from the published examples, as discussed in Sec.~\ref{sec:implementation}.

\begin{lstlisting}[language=Julia]
rng = Random.MersenneTwister(9790242)
slow_and_short = all_pairs(parameters...; engine = GND(rng = rng, M = 50))
\end{lstlisting}

\subsection{Evaluate results}

When a testing suite contains only a few tests of a function, the test author can usually figure out how to check those few function outputs for failures. Automatic generation of test arguments creates a problem because it isn't paired with automatic generation of checks for test failures.

\vskip 6pt
One solution is to create a parallel implementation of the function under test. It could be an earlier version of the function, using a na{\"i}ve algorithm. It could be a function that computes the same values using a different mathematical representation, such as using numerical integration instead of using the result of symbolic integration. It could be an external implementation, in another language.

\vskip 6pt
It helps to consider the check for failure, not as \verb|result == oracle(test_case...)| but as a rule that takes both result and test case into account. For instance, some mathematical calculations solve inverse problems, so that the result can be fed into a related forward problem. The test becomes whether it can recover its input values.

\vskip 6pt
Automated testing of results often takes the form of testing for traits of the result, not exact values. For example, you could test a function for continuity of outputs near the given input. You could use statistical measures, such as maximum absolute error, to assert that a function's output is near some standard result. There is a powerful technique to assert symmetries in function arguments. For instance if, for some function under test, $f$, a symmetry test might check that \verb|f(a, b)=f(2*a, b/2)|~\cite{Segura2016-qh}.


\section{Optimization of Test Cases with Practical Constraints}\label{sec:implementation}

There are two ways to improve upon current use of combinatorial test generation. One is to help the tester whose testing resources are limited, by generating fewer test cases that still meet the wayness guarantee. The other is to make it more convenient to use combinatorial testing within the unit testing framework. Here, we could generate tests with so little time and memory that there is no need to cache test cases. Both of these improvements are limited by our need for algorithms that support the features, described above, that make automated test case generation practical to use.

\vskip 6pt
The \utd package expanded on the implementation of two $n$-way combinatorial interaction testing algorithms. One is the greedy, non-deterministic algorithm known as \textsc{aetg}~\cite{Cohen1997-lb}. The other is a deterministic, parameter-choosing algorithm called \textsc{ipo-g}~\cite{Lei2008-xt}. Like many greedy algorithms, these can be sensitive to small, sometimes unspecified, choices in implementation. We used two approaches for this package in order to make algorithms that had the right features and performed well.

\vskip 6pt
In order to support a feature where the test author can require certain pre-defined test cases, we added those test cases to the list of tuples to be covered, where a tuple is a combination of argument values. We just put that into the list and let the algorithms take care of covering it.

\vskip 6pt
In order to support multi-level wayness in coverage, we used multiple rounds of the official algorithm, starting at the highest wayness. For example, given a pairwise test generation strategy that requests all-triples over a set of five arguments, we first generated test cases that cover the all-triples, and then we presented these as initial seed values to the all-pairs calculation.

\vskip 6pt
In order to avoid forbidden combinations of arguments, we checked the validity of the next possible argument at every moment in the algorithm that a next argument is chosen. This sometimes meant that a test case being created could be thrown out and started new from a different seed argument value. This is a situation where it might help to use the Z3 theorem prover to reduce the need for retries, as some packages have done~\cite{Nie2011-yl}.

\vskip 6pt
There are a lot of choices in how to implement these algorithms, and they matter for speed and efficiency. The most important strategy for designing these algorithms was to generate an ensemble of algorithms and test them against sample inputs.

\vskip 6pt
For example, there is a common moment in these algorithms where a putative test case, partially specified, is matched against a coverage tuple, which will have $n$ specified argument values for $n$-way coverage. Given just one argument from the test case and one argument from the tuple, there are five ways they can compare.
\begin{center}
\begin{tabular}{rll}
comparison & test case & tuple \\ \hline
ignores & * & * \\
skips & $a$ & * \\
misses & * & $b$ \\
matches & $a$ & $b=a$ \\
mismatches & $a$ & $b\ne a$
\end{tabular}
\end{center}
Only the last comparison is clearly a mismatch. If we compare multiple arguments of the test case and tuple, we can quantify that comparison as having zero or more ``ignores,'' zero or more ``skips,'' and so-on. This means there are $2^4-1$ ways that a non-zero test case and tuple can compare, even if they have no mismatches.

\vskip 6pt
The \textsc{ipo-g} algorithm matches tuples in two places, once when it expands the number of columns of test cases and once when it fills in values at the end. In both cases, after testing $15\times 15$ ways of matching, we found that the simple requirement that there be no mismatches performed as well as any other choice, while more complicated matching strategies resulted in significantly more test cases for the same coverage.

\vskip 6pt
There are plenty of improvements to make to these algorithms, such as the use of advanced data structures~\cite{Segall2011-jv} or more adaptive optimization techniques like simulated annealing~\cite{Petke2015-ex}. There are also completely different greedy algorithms~\cite{Calvagna2012-ic,Koc2018-vs}. Any of these would be welcome to serve some use case if they can remain convenient to use within the testing framework.


\section{Comparison of Approaches}\label{sec:comparison}

Now that we've explored combinatorial interaction testing, let's return to place it in the context of other similar testing techniques. Compared to other methods for automated test case generation, combinatorial interaction testing makes fewer tests, with some wayness guarantees.

\vskip 6pt
Random testing also generates test cases, and we know, from Arcuri et al, that these test cases increase code coverage with increasing numbers of test cases, in a predictable way~\cite{Arcuri2012-az}. Random testing also has the benefit that you can run it for an arbitrarily long time during acceptance testing. There is also no requirement that values be completely random within the domain of arguments. They are almost always biased towards values that carry risk, such as empty strings or other corner cases. Here, a preference for combinatorial interaction testing comes from a lack of test-running resources or the ease of specifying sample argument values instead of the complete domain of the arguments.

\vskip 6pt
There are also more mathematical test generation methods, such as orthogonal arrays or Sobol sequences~\cite{He2013-th}. These give an even distribution of values within high-dimensional spaces. They would be excellent to, for instance, seed an optimization problem with local minima, but they aren't designed to explore the space of execution paths of code, in the way that combinatorial methods are.

\vskip 6pt
There are also powerful testing methods that combine automated generation of test inputs with observation of test execution and outputs. For tests that run reasonably quickly, these strategies can shorten the time to find faults in the code, even if they don't shorten the length of a particular test run.

\vskip 6pt
Property-based testing offers domain-specific ways to describe the domain of each argument, or sets of arguments, so that the test author can focus on the domain of the function instead of the exact test suite. Tools of this kind can watch for failed tests, and then they can simplify the test arguments while checking for failure, in order to find the simplest test that fails. This is extraordinarily useful for fault-finding but, again, will use more testing resources~\cite{loscher2018automating}.

\vskip 6pt
Concolic testing performs the remarkable feat of executing a test case and recording that execution in order to learn which argument values control choices at each decision in the code. Then the concolic algorithm chooses a different argument value and tries the test again. This is an automated way to learn corner cases in the code, which should help with the challenge of deciding equivalence classes for arguments. At the same time, concolic testing generally uses logic solvers to decide equivalence classes, and these can be idiosyncratic for mathematical code~\cite{King1976-jt,Wang2018-xh}. The hardest part about concolic testing may be running it, conveniently, within the unit testing framework.

\section{Conclusion}
A lot of the strategies for using combinatorial testing, described in Sec.~\ref{sec:how-to-use}, also apply to competing methods for automated test case generation in Sec.~\ref{sec:comparison}. We showed that these methods apply to many kinds of software testing, from unit testing of functions to user-level testing of whole applications. We call them automated because the competing method is assiduous, manual creation of sets of unit tests. So automation is clearly an improvement.

\vskip 6pt
By showing how to use combinatorial interaction testing, we have shown that it should be used. This applies to other test generation techniques, too.

\vskip 6pt
Residual logical complexity.



\vskip 6pt
This would work well with automated test selection if the the Julia testing framework defined tests such that they could be selected, as is done in X package.


\vskip 6pt
Unit test is a second telling of a function. A parable.

\input{bib.tex}

\end{document}

% Inspired by the International Journal of Computer Applications template
